\documentclass{article}

\usepackage[a4paper,left=18mm,right=18mm,top=20mm,bottom=18mm]{geometry}
\usepackage[italian]{babel}

\usepackage{titling}
\usepackage{graphicx}

\title{Report analisi dell'aria}
\author{A}
\date{\today}

\makeatletter         
\def\@maketitle{
\noindent\begin{minipage}{\textwidth}
\begin{minipage}[c]{0.8\textwidth}
\begin{center}
{\Huge \bfseries \sffamily \@title }\\[4ex] 
{\Large  \@author}\\[4ex] 
\@date\\[0ex]
\end{center}
\end{minipage}\hfill
\begin{minipage}[c]{0.2\textwidth}
\raggedleft
\includegraphics[width = 20mm]{logo.png}\\[0.1ex]
\includegraphics[width = 25mm]{UniBg-logo.jpg}
\end{minipage}
\end{minipage}
}
\makeatother


\begin{document}

\maketitle

\par\noindent\rule{\textwidth}{0.4pt}
\textbf{Abstract:}
\par\noindent\rule{\textwidth}{0.4pt}

\section{Introduzione}
La fase iniziale del progetto consiste nell'analisi nell'arco temporale 2018-2020 dei dati 
forniti dal sito ARPAL relativi allo studio del $NH_{3}$ e dei particolati atmosferici $PM_{10}$ e $PM_{2.5}$ al 
fine di dimostrare i risultati positivi e la diminuzione dell'inquinamento dell'aria 
dovuti dal Covid19 in Lombardia.

\section{Analisi dei dati}
La fase iniziale del progetto consiste nel cercare le centraline in Lombardia
che misurano contemporaneamente $NH_{3}$, $PM_{10}$ e $PM_{2.5}$ oppure solo due di essi
(sono ammessi dei dati mancanti sporadicamente).
\end{document}
